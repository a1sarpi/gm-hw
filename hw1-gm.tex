\documentclass[12pt]{article}
\usepackage{myHeadArt}
\usepackage{mathrsfs}
\renewcommand{\qedsymbol}{$\blacksquare$}

\usepackage{float}
\usepackage{listings}
\usepackage{minted}
\usepackage{xfrac}

\usepackage{geometry}
\geometry{verbose,a4paper,tmargin=1cm,bmargin=2cm,lmargin=1.5cm,rmargin=1.5cm}

% \usepackage{svg}
\usepackage{color}
\definecolor{lightgray}{rgb}{.9,.9,.9}
\definecolor{darkgray}{rgb}{.4,.4,.4}
\definecolor{purple}{rgb}{0.65, 0.12, 0.82}

\lstdefinelanguage{JavaScript}{
  keywords={typeof, new, true, false, catch, function, return, null, catch, switch, var, if, in, while, do, else, case, break},
  keywordstyle=\color{blue}\bfseries,
  ndkeywords={class, export, boolean, throw, implements, import, this},
  ndkeywordstyle=\color{darkgray}\bfseries,
  identifierstyle=\color{black},
  sensitive=false,
  comment=[l]{//},
  morecomment=[s]{/*}{*/},
  commentstyle=\color{purple}\ttfamily,
  stringstyle=\color{red}\ttfamily,
  morestring=[b]',
  morestring=[b]"
}

\lstset{
   language=JavaScript,
   backgroundcolor=\color{lightgray},
   extendedchars=true,
   basicstyle=\footnotesize\ttfamily,
   showstringspaces=false,
   showspaces=false,
   numbers=left,
   numberstyle=\footnotesize,
   numbersep=9pt,
   tabsize=2,
   breaklines=true,
   showtabs=false,
   captionpos=b
}


\begin{document}
\pagestyle{empty}
\centerline{\large Министерство науки и высшего образования}  
\centerline{\large Федеральное государственное бюджетное образовательное}
\centerline{\large учреждение высшего образования}
\centerline{\large ``Московский государственный технический университет}
\centerline{\large имени Н.Э. Баумана}
\centerline{\large (национальный исследовательский университет)''}
\centerline{\large (МГТУ им. Н.Э. Баумана)}
\hrule
\vspace{0.5cm}
\begin{figure}[h]
\center
\includegraphics[height=0.35\linewidth]{bmstu-logo.pdf}
\end{figure}
\begin{center}
  \large  
  \begin{tabular}{c}
    Факультет ``Фундаментальные науки'' \\
    Кафедра ``Математика и компьютерные науки (ФН11)''    
  \end{tabular}
\end{center}
\vspace{0.5cm}
\begin{center}
  \LARGE \bf  
  \begin{tabular}{c}
    \textsc{Геометрическое моделирование} \\
    Домашнее задание №1 \\
  \end{tabular}
\end{center}
\vspace{0.5cm}
\begin{center}
  \large  
  \begin{tabular}{c}
    Клячко Вячеслав (ФН11-61Б) \\ % sasha
    Вариант 8 \\ \\ % sasha: 9
    Преподаватель: Захаров А.\,Н.
  \end{tabular}
\end{center}
\vfill
\begin{center}
  \large  
  \begin{tabular}{c}
    Москва, 
    2024 г.
  \end{tabular}
\end{center}
\pagebreak



\section*{Условие}
Напишите программу построения естественного бикубического сплайна по контрольным
точкам, лежащим на цилиндрической поверхности заданного радиуса. Используйте
краевые условия третьего типа по угловой координате и 
первого % sasha: пятого
типа по параметрической координате вдоль оси цилиндра (для их задания используйте
разностные аппроксимации). Используйте программу л. р. №2 (Примечание: в функции
\texttt{generateControlPoints}, в которой генерируются координаты контрольных точек, используйте параметрические уравнения для цилиндрической поверхности заданного радиуса).

\section*{Решение}
\subsection*{Введение параметрических координат}
Дана декартова система координат $ Oxyz $, радиус и $ x $-координаты двух
оснований цилиндра: $ x_{\min} $ и $ x_{\max} $.

\begin{figure}[H]
  \centering
  \includegraphics[width=0.8\textwidth]{Figures/1bis.png}
  \caption{}
  \label{fig:1}
\end{figure}
Этих данных хватает для нахождения всех точек цилиндра. Найдём координаты
некоторых из них, которые далее будем называть \emph{контрольными}. Для этого в функции \texttt{generateControlPoints}, взятой из
л. р. №2, добавим цикл
\begin{lstlisting}[caption=Создание контрольных точек по входным параметрам.]
for (let i = 0; i < this.N_ctr; i++)
    for (let j = 0; j < this.M_ctr; j++) {
        const x = Xmin + i * (Xmax - Xmin) / (this.N_ctr - 1) - this.Xmid;
        const y = Radius * Math.cos( 2 * j * Math.PI / (this.M_ctr) );
        const z = -Radius * Math.sin( 2 * j * Math.PI / (this.M_ctr) );

        this.add_coords(i, j, x, y, z);
    }
\end{lstlisting}

Здесь неявно были введены стандартные для цилиндра параметрические координаты $
\varphi$, $ h $. Число
$ \varphi \in [0, 2\pi)$ соответствует углу, отсчитаемого от вертикальной оси $ Oy $, под которым находится
проекция точки на основание; а $ h \in [0, x_{\max} - x_{\min}]$ есть
расстояние до основания (на рисунке $ \ref{fig:1} $ расположенного слева). По циклу видно, что контрольные точки выбираются равномерно по каждой
параметрической координате в пределах их изменения. Число \verb|N_ctr| контрольных точек вдоль оси
цилиндра и число \verb|M_ctr| контрольных точек по сечению цилиндра задаётся
пользователем с помощью ползунка.

Итак, нами был получен набор декартовых координат контрольных точек (обозначим
их $ p_{ij} $, где $ i $ соответствует координате $ h $, а $ j $ --- координате
$ \varphi $), которые
далее мы должны интерполировать с помощью естественного бикубического сплайна.
Для начала введём другие параметрические координаты $ u ,  v \in [0, 1] $ тремя
известными из второй лабораторной работы способами и найдём эти координаты для
контрольных точек. Заметим, что при фиксированном индексе $ i $ все координаты
$ v(p_{ij}) = v_i $ равны между собой, ровно как и координаты $ u(p_{ij}) = u_j $ при
фиксированном индексе $ j $.


\subsection*{Введение естественного бикубического сплайна}
Для интерполяции воспользуемся формулой естественного бикубического сплайна 
\[
  S(u, v) = \sum_{k=0}^3\sum_{l=0}^3 a^{ij}_{kl} (u-u_j)^k(v - v_i)^l =
  S_{ij}(u, v),
\]
справедливой внутри каждой ячейки $ \Omega_{ij} $, то есть при 
\[
  (u, v) \in \Omega_{ij} = \{(u,v) \mid u_j \leqslant u \leqslant u_{j+1}, \,
    v_i \leqslant
  v \leqslant v_{i+1}\}.
\]
Коэффициенты $ a^{ij}_{kl} $ неизвестны, и наша задача сводится к тому, чтобы их
найти.

Для удобства введём для каждой ячейки вспомогательные координаты $ \omega, \xi
\in [0, 1]$,
соответствующие прежним координатам $ u $, $ v $ с той разницей, что в данной
ячейке они изменяются от нуля до единицы. Находясь в ячейке $ \Omega_{ij} $, это можно сделать, например, по
формулам перехода
\[
  \omega = \frac{u - u_j}{u_{j+1} - u_j}, \qquad \xi = \frac{v - v_i}{v_{i+1} -
  v_i}.
\]
Формула для сплайна (пусть и с известным образом изменёнными коэффициентами)
тогда примет совсем простой вид
\[
  S(\omega, \xi) = \sum_{k=0}^3\sum_{l=0}^3 a^{ij}_{kl} \omega^k\xi^l.
\]

% Понятно, что (учитывая требование $ S \in C^{2,2}[\Omega] $) $ S(0, 0) = a^{ij}_{00} = p_{ij} $, только тогда сплайн
% интерполяционный. 

\subsection*{Составление СЛАУ}
Запишем наши требования к сплайну.
\begin{enumerate}
  \setcounter{enumi}{-1}
  \item Сплайн имет непрерывные частные и смешанные производные до второго
    порядка включительно: $ S \in C^{2,2}[\Omega] $.
  \item Для того чтобы сплайн был интерполяционным, необходимо выполнение
    условий
    \[
      S_{ij}(0, 0) = a_{00}^{ij} = p_{ij}, \quad
      S_{ij}(0, 1) = \sum_{l=0}^3 a_{0l}^{ij} = p_{i+1,j}, \quad
      S_{ij}(1, 0) = \sum_{k=0}^3 a_{k0}^{ij} = p_{i, j+1}.
    \]
    % , $ S_{ij}(0, 1) = p_{i+1,j} $, $ S_{ij}(1,
    % 0) = p_{i, j+1} $.
  \item Для стыковки частных производных  
  \begin{align*}
    \frac{\partial S_{ij}}{\partial \omega}(\omega, \xi) &=
    \sum_{k=1}^3\sum_{l=0}^3 k a_{kl}^{ij}\omega^{k-1}\xi^l,\\
    \frac{\partial^2 S_{ij}}{\partial \omega^2}(\omega, \xi) &=
    \sum_{k=2}^3\sum_{l=0}^3 k(k-1)a_{kl}^{ij}\omega^{k-2}\xi^l
  \end{align*}
  необходимо выполнение условий
  \begin{align*}
    \frac{\partial S_{ij}}{\partial \omega}(1, 0) &=  \frac{\partial S_{i, j+1}}{\partial
    \omega}(0,0), & \text{то есть }\sum_{k=1}^3 ka^{ij}_{k0} &= a_{10}^{i,j+1},\\   
    \frac{\partial^2 S_{ij}}{\partial \omega^2}(1, 0) &=  \frac{\partial^2 S_{i,
    j+1}}{\partial
      \omega^2}(0,0), & \text{то есть }\sum_{k=2}^3
      k(k-1)a_{k0}^{ij} &= 2a^{i,j+1}_{20}.
  \end{align*}
\item Для стыковки частных производных  
\begin{align*}
  \frac{\partial S_{ij}}{\partial \xi}(\omega, \xi) &=
  \sum_{k=0}^3\sum_{l=1}^3 l a_{kl}^{ij}\omega^{k}\xi^{l-1},\\
  \frac{\partial^2 S_{ij}}{\partial \xi^2}(\omega, \xi) &=
  \sum_{k=0}^3\sum_{l=2}^3 l(l-1) a_{kl}^{ij}\omega^{k}\xi^{l-2}
%   \frac{\partial S_{ij}}{\partial \xi}(\omega, \xi) &= ,\\ 
%   \frac{\partial^2 S_{ij}}{\partial \xi^2}(\omega, \xi) &= 
\end{align*}
необходимо выполнение условий
  \begin{align*}
    \frac{\partial S_{ij}}{\partial \xi}(0, 1) &= \frac{\partial S_{i+1, j}}{\partial
    \xi}(0,0), & \text{то есть } \sum_{l=1}^3 l a_{0l}^{ij} &= a_{01}^{i, j+1},\\
      \frac{\partial^2 S_{ij}}{\partial \xi^2}(0, 1) &= \frac{\partial^2 S_{i+1,
    j}}{\partial
        \xi^2}(0,0), & \text{то есть } \sum_{l=2}^3 l(l-1) a_{0l}^{ij} &=
        2a_{02}^{i, j+1}.
  \end{align*}
\item Наконец, для стыковки смешанных производных 
\[
  \frac{\partial^2 S_{ij}}{\partial \omega \partial \xi} = \sum_{k=1}^3
  \sum_{l=1}^3 kl a^{ij}_{kl}\omega^{k-1}\xi^{l-1}
\]
необходимо выполнение условий
%TODO
\item Полученная система линейных уравнений не определена, поэтому потребуются
  граничные условия. В задаче требуются использовать смешанные граничные
  условия, третьего типа (периодические)
  по $ u $ и 
  первого типа % sasha: пятого типа
  по $ v $.
  \begin{enumerate}
    \item Периодические условия требуют также склеивания точек с $ p_{iM} $ с
      точками $ p_{i1} $. Отсюда  
      \begin{align*}
        \frac{\partial S_{iM}}{\partial \omega}(1, 0) &= \frac{\partial S_{i,
        1}}{\partial
          \omega}(0,0), & \text{то есть } \sum_{k=1}^3ka^{iM}_{k0} &=
          a_{10}^{i1},\\
    \frac{\partial^2 S_{iM}}{\partial \omega^2}(1, 0) &= \frac{\partial^2 S_{i, 1}}{\partial
    \omega^2}(0,0), & \text{то есть } \sum_{k=2}^3 k(k-1) a^{iM}_{k0} &=
    2a_{20}^{i1}.
      \end{align*}
      %TODO: смешанные?
     
    \item % sasha
      В граничных и угловых точках сетки зададим также условия первого типа: 
      \[
        S^{(0, 1)}(0, 0) = p_{ij}^{(0, 1)},\qquad i\in\{0, N\}, \ j =
        0,\ldots,M.
        % S^{(1,1)}(0,0) &= p_{ij}^{(1,1)}, \qquad i \in \{0, N\}, \ j\in\{0,M\}.
      \]
      Найдём значения величин $ p_{0j}^{(0, 1)} $, $ p_{Nj}^{(0,1)} $ для цилиндра.
      %TODO: значения pij'
  \end{enumerate}
\end{enumerate}

%TODO: таблица коэффициентов ?

\subsection*{Решение полученной СЛАУ}
Итак, мы получили СЛАУ, которую будем решать методом Жордана --
Гаусса\footnote{Не будь краевых условий, мы имели бы трёхдиагональную матрицу
коэффициентов и смогли бы воспользоваться методом прогонки.}:
\begin{lstlisting}
function zeros(M, n, m) {
    for(let i = 0; i < n; ++i)
        for (let j = 0; j < m; ++j)
            M[i][j] = 0.;
}

function diagonalize(M) {
    let m = M.length;
    let n = M[0].length;
    for(let k=0; k<Math.min(m,n); ++k) {
        // Find the k-th pivot
        i_max = findPivot(M, k);
        if (M[i_max, k] == 0)
            throw "matrix is singular";
        swap_rows(M, k, i_max);
        // Do for all rows below pivot
        for(let i=k+1; i<m; ++i) {
            // Do for all remaining elements in current row:
            let c = M[i][k] / M[k][k];
            for(let j=k+1; j<n; ++j) {
                M[i][j] = M[i][j] - M[k][j] * c;
            }
            // Fill lower triangular matrix with zeros
            M[i][k] = 0;
        }
    }
}

function findPivot(M, k) {
    let i_max = k;
    for(let i=k+1; i<M.length; ++i)
        if (Math.abs(M[i][k]) > Math.abs(M[i_max][k]))
            i_max = i;
    return i_max;
}

function swap_rows(M, i_max, k) {
    if (i_max != k) {
        let temp = M[i_max];
        M[i_max] = M[k];
        M[k] = temp;
    }
}

function makeM(A, b) {
    let M = new Array(A.length);
    for (let i = 0; i < A.length; ++i)
        M[i] = new Array(A[i].length + 1);
    for (let i = 0; i < A.length; ++i) {
        for (let j = 0; j < A[i].length; ++j)
            M[i][j] = A[i][j];
        M[i][A[i].length] = b[i];
    }
    return M;
}

function substitute(M) {
  let m = M.length;
  for(let i=m-1; i>=0; --i) {
    let x = M[i][m] / M[i][i];
    for(let j=i-1; j>=0; --j) {
      M[j][m] -= x * M[j][i];
      M[j][i] = 0;
    }
    M[i][m] = x;
    M[i][i] = 1;
  }
}

function extractX(M) {
  let x = [];
  let m = M.length;
  let n = M[0].length;
  for(let i=0; i<m; ++i){
    x.push(M[i][n-1]);
  }
  return x;
}

function solve(A, b) {
    let M = makeM(A,b);
    diagonalize(M);
    substitute(M);
    let x = extractX(M);
    return x;
}
\end{lstlisting}


\subsection*{Демонстрация программы}
%TODO: куча картинок



\end{document}
